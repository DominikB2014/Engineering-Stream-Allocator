\documentclass[12pt]{article}

\usepackage{graphicx}
\usepackage{paralist}
\usepackage{listings}
\usepackage{booktabs}
\usepackage{indentfirst}

\oddsidemargin 0mm
\evensidemargin 0mm
\textwidth 160mm
\textheight 200mm

\pagestyle {plain}
\pagenumbering{arabic}

\newcounter{stepnum}

\title{Assignment 2}
\author{Dominik Buszowiecki - buszowid}
\date{\today}

\begin {document}

\maketitle

The purpose of this software design exercise is to write a Python program responsible for allocating first-year
engineering students at McMaster University, into their second-year program using a formal specification.
The program consists of the following files: AALst.py, DCapALst.py, Read.py, SALst.py, SeqADT.py, StdntAllocTypes.py,
test\_All.py, and Makefile.

\section{Testing of the Original Program}

The approach taken to test the required modules was to ensure that every line of code is correct and works as intended.
To assist with this goal, the pytest coverage option was used to see the percentage of lines covered in every module.
Due to the fact that most of the methods were quite small, it was easy to determine what test cases should be performed
for each method.

There was a total of 30 test performed, the tests and their rational are written below.
\medskip

\textbf{SeqADT.py:}
\begin{itemize}

  \item SeqADT.\_\_init\_\_
    \begin{enumerate}
      \item test\_empty: Ensures that when we construct a object of type SeqADT using an empty list,
      the object also contains an empty list.
    \end{enumerate}

  \item SeqADT.start
    \begin{enumerate}
      \item test\_start: Ensures that when we run the start method, the index state variable gets reset to 0.
    \end{enumerate}

  \item SeqADT.next
    \begin{enumerate}
      \item test\_next: Tests that the next item in the sequence is correct.
      \item test\_stop\_iteration: Checks if StopIteration is raised when there are no more items in the sequence.
    \end{enumerate}

  \item SeqADT.end
    \begin{enumerate}
      \item test\_end\_false: Tests if the end method returns False when there are items left in the sequence.
      \item test\_end\_true: Tests if the end method return True if there are no items left in the sequence.
    \end{enumerate}

\end{itemize}
\medskip

\textbf{DCapALst.py:}
\begin{itemize}

  \item DCapALst.init
    \begin{enumerate}
      \item test\_constructor: Tests if the constructor successfully creates an empty list.
    \end{enumerate}

  \item DCapALst.add
    \begin{enumerate}
      \item test\_add: Ensures that when we add a department and its capacity, it appears in DCapAList.
      \item test\_add\_exception: Tests if a KeyError is raised when we attempt to add a department that has been added
      already.
    \end{enumerate}

  \item DCapALst.remove
    \begin{enumerate}
      \item test\_remove: Ensures that a department can be successfully removed from DCapALst.
      \item test\_remove\_exception: Ensures that the remove method raises a KeyError if the department it is trying
      to remove is not in DCapALst.
    \end{enumerate}

  \item DCapALst.elm
    \begin{enumerate}
      \item test\_elm\_true: Tests that the elm method returns True if a department has already been added.
      \item test\_elm\_false: Tests that the elm method returns False if a department has not been added.
    \end{enumerate}

  \item DCapALst.capacity
    \begin{enumerate}
      \item test\_capacity: Ensures that the capacity method returns the correct capacity of a particular department.
      \item test\_capacity\_exception: Tests if the capacity method raises a KeyError if the department given is
      not in DCapALst.
    \end{enumerate}

\end{itemize}
\medskip

\textbf{SALst.py:}
\begin{itemize}

  \item SALst.init
    \begin{enumerate}
      \item test\_constructor: Tests that the constructor successfully creates a empty list.
    \end{enumerate}

  \item SALst.add
    \begin{enumerate}
      \item test\_add: Tests that the add method successfully adds a students macid and their information into the
      student list (SALst).
      \item test\_add\_exception: Ensures that the add method raises a KeyError when we try to add a student that has
      already been added.
    \end{enumerate}

  \item SALst.remove
    \begin{enumerate}
      \item test\_remove: Ensures a student can be successfully deleted from SALst.
      \item test\_remove\_exception: Tests that the remove method raises a KeyError when we try to remove a student
      that is not in SALst.
    \end{enumerate}

  \item SALst.elm
    \begin{enumerate}
      \item test\_elm\_true: Tests that the elm method returns True when a student is in SALst.
      \item test\_elm\_false: Tests that the elm method returns False when a student is not in SALst.
    \end{enumerate}

  \item SALst.info
    \begin{enumerate}
      \item test\_info: Ensures that the info method returns the correct information of a particular student.
      \item test\_info\_exception: Test that the info method raises a KeyError if we try to retrieve information of
      a student that is not in SALst.
    \end{enumerate}

  \item SALst.sort
    \begin{enumerate}
      \item test\_sort\_empty: Tests that the sort method is able to return a empty list.
      \item test\_sort: Tests that the sort method is able to read the filtering function correctly and sort only
      the students who pass the filter by GPA.
    \end{enumerate}

  \item SALst.average
    \begin{enumerate}
      \item test\_average: Tests that the average method can read the filtering function correctly and compute the
      average amongst the students who pass the filter.
      \item test\_average\_exception: Ensures that a ValueError is raised when we try to compute the average between
      0 students (Due to no students passing the filter).
    \end{enumerate}

  \item SALst.allocate
    \begin{enumerate}
      \item test\_allocate\_normal: A normal case for allocating students where some students: have free choice,
      get their first choice, and do not get their first choice.
      This test case ensures that students are allocated into the correct program.
      \item test\_allocate\_exception: Ensures that the allocate method will raise a RuntimeError if a student does not
      get into any of their choices because all their choices are full.
    \end{enumerate}

\end{itemize}
\medskip

In the end, all test cases passed without any issues and pytest showed 100\% coverage for all tested modules.
However, I did discover a minor issue during the first round of testing.
The issue occurred in the remove method of DCapALst.py, the issue was forgetting to add a return after I removed the
student from the list.
The lack of a return meant that the method would run the remaining lines of code after the deletion, therefore
always raising a KeyError no matter what the input was.
The problem however, was easy to debug because the modular design of the program allowed me to pinpoint it with ease.

\section{Results of Testing Partner's Code}

  After running the test cases on my partner's code, all but one passed.

  By looking through my partner's code, it was clear why there was such a high rate of success.
  This is because my partner's code was very similar (almost identical) to mine.
  The reason for the similarities is due to the use of a formal design specification which when understood correctly,
  can only be interpreted in one way.

  The test case that failed was test\_add in DCapALst.
  The reason for failure is due to the fact the my partner used dictionaries whereas I used a list of tuples to
  represent a department and its capacity.
  Therefore when test\_add accessed the state variable directly, it expected a list and not a dictionary.
  With a minor modification to each method in DCapALst, I was able to get all the test cases to pass.
  The modification done was changing the state variable to a list and changing the indexing in each method to the one
  used with lists instead of dictionaries.

\section{Critique of Given Design Specification}

  The thing I enjoyed about the given design specification is that it was not very ambiguous.
  The reason for this is due to the use of a formal design specification which uses mathematical notation to
  precisely define the program.
  The lack of ambiguity provided me with a greater degree of confidence that my implementation was correct.

  What I disliked about the design is that I felt as though it was much less interesting to program in comparison to A1.
  The program was broken up into many small pieces that were defined formally making it obvious on the way
  it should be coded.
  This made me feel less creative as many design decisions were not imposed by me.

  Another thing I did not like is reading the math notation in order to understand what a particular method should do.
  It seems backwards to me that we read the notation, convert it back to a natural language (in our mind) and
  once again write it formally into python.
  I also felt as though reading the math notation took much longer to understand than reading the natural language
  version in A1.

  In order to improve the specification, my suggestion would be to somehow combine the natural design of A1 with the formal
  design of A2.
  Although it would likely take longer to write such a specification, it would be both less ambiguous and quicker to
  understand.

\section{Answers}


\begin{enumerate}[a)]

  \item
  The main advantage with the use a formal design specification is the lack of ambiguity for the programmer.
  The lack of ambiguity results in less confusion and questions regarding the implementation.
  As mentioned before, when understood correctly, a formal design specification should only be interpreted in one way.
  The same cannot be said about a natural language specification
  This will give the one who created the specification a great deal of assurance that the program will work as
  he intended it to.

  One of the disadvantages that I see in a formal design specification is the background knowledge required to
  understand what is being asked of you.
  In order to read a MIS, you require some knowledge on discrete mathematics in order to determine what a particular
  method should do.
  Even if you have this knowledge, it could take a while to convert the given notation into something more
  understandable.
  If the specification used natural language, it would be easier to understand what the program should do.

  Another disadvantage to the given design specification is the lack of creativity allowed for the programmer.
  The specification is given in such a precisely defined way that the programmer often doesn't have to
  come up with their own algorithms.
  Instead, they just follow exactly what the specification says.
  In a natural language specification, it is often up to the programmer to come up with the algorithms as well as
  make their own design decisions.

  \item In my opinion, the best way to change the assumption that the GPA is between 0 and 12 to an exception is to
  modify the specification for the Read module.
  The change required would be checking if the GPA is in the correct range before adding a student to SALst in the
  load\_stdnt\_data method.
  If the student is not in the correct range, we simply raise an exception.
  The reason I believe this is the best way is because when the program is actually used, the user would first have to
  load student data from a file.
  By adding the exception in the Read module, we catch a mistake early and prevent it from carrying into the rest of the
  program.

  If the exception is incorporated in Read, the specification would not require you to replace a record type with a new
  ADT. This is because we raise the exception before we even add a student into the record type rather than inside
  the type.

  \item If we ignore sort, average, and allocate in SALst and DCapALst, the modification required would be
  incorporating some form of inheritance between the classes.
  This could be done several ways, such as creating a general list class and having SALst and DCapALst inherit list.
  The list class would need to contain methods that both SALst and DCapALst share in common such as init, remove,
  and elm.
  Then, when we create DCapALst and SALst we only need to add on the methods that are specific to these classes such as
  capacity for DCapALst and allocate, average, sort and info for SALst.

  \item A2 is more general than A1 in a few ways.
  One of the main ways this can be observed is in the average and sort function of the SALst module.

  In A1, the sort function returned a list of every student sorted by their GPA.
  However, in A2 the sort method is able to return a particular subset of students sorted by their GPA.
  This subset can be specified by the programmer.

  In A1, the average function was only able to compute the average amongst a particular gender whereas in A2,
  we are able to define what subset of students we want to compute the average amongst.

  \item By using SeqADT instead of a regular list, we provide level encapsulation.
  Encapsulation means hiding the internal information of one piece of code from another.
  By using SeqADT, we are not able to access the list of choices directly, instead we must access it through the
  methods of the SeqADT class.
  The advantage to encapsulation is that you can be more confident that accidental changes cannot be made to the
  list as only acceptable changes defined by your methods are allowed.

    Another advantage to using SeqADT rather than a list is it makes other functions easier to program and read.
  When we create a abstract data type like SeqADT, we often model it based on real world things.
  In this case we are modelling SeqADT after a sequence of values.
  This makes it much easier to program and read because the code will look closer to natural language.

  \item The advantage of using Enums is to know exactly what the possible options are for a particular field.
  For example, if we wanted to know the possible engineering choices, we simply look at the documentation for
  StdntAllocTypes and observe the possible values.
  We also do not have to worry about spelling as the spelling is precisely defined.
  The fact that we don't have to worry about spelling benefits the programmer as they will likely encounter far less
  problems related to grammar.

  The reason why Enums were not introduced in the specification for macids is because there are too many
  possible macid's.
  Remember, a Enum is able to map a particular string to a value, therefore we would need to have a enum for every
  macid.
  Every year there are new engineering students, therefore every year we would be required to change the enums for
  every macid.
  By knowing this, it would be much easier to simply use strings as Enums would make the software harder to maintain.

\end{enumerate}

\newpage

\lstset{language=Python, basicstyle=\tiny, breaklines=true, showspaces=false,
  showstringspaces=false, breakatwhitespace=true}
%\lstset{language=C,linewidth=.94\textwidth,xleftmargin=1.1cm}

\def\thesection{\Alph{section}}

\section{Code for StdntAllocTypes.py}

\noindent \lstinputlisting{../src/StdntAllocTypes.py}

\newpage

\section{Code for SeqADT.py}

\noindent \lstinputlisting{../src/SeqADT.py}

\newpage

\section{Code for DCapALst.py}

\noindent \lstinputlisting{../src/DCapALst.py}

\newpage

\section{Code for AALst.py}

\noindent \lstinputlisting{../src/AALst.py}

\newpage

\section{Code for SALst.py}

\noindent \lstinputlisting{../src/SALst.py}

\newpage

\section{Code for test_All.py}

\noindent \lstinputlisting{../src/test_All.py}

\newpage

\section{Code for Read.py}

\noindent \lstinputlisting{../src/Read.py}

\newpage

\section{Code for Partner's SeqADT.py}

\noindent \lstinputlisting{../partner/SeqADT.py}

\newpage

\section{Code for Partner's DCapALst.py}

\noindent \lstinputlisting{../partner/DCapALst.py}

\newpage

\section{Code for Partner's SALst.py}

\noindent \lstinputlisting{../partner/SALst.py}

\end {document}